\section{Problem Set 9}

To do:
- integration of navigation and control (Tycho) - DONE
- implement drag in ground truth propagation (Tycho) - DONE
- fix geometric mapping (Tycho) - DONE

- implement reachable set impulsive control (maybe don't need?)
- try double radial burns for clean control in mode 3-4 (Anshuk)
- three tangential burn method for control (further away, more efficient)
+ clean up control stuff (Anshuk)
- look into min delta v for continuous control (Anshuk)



- make report cohesive 
- incorporate all feedback

- add SRP and third body effects to ground truth propagation? (Tycho)
- model sensors and actuators
    - better representation of GPS and vision?


Stuff to remember: 
6‑unit CubeSat (6 U), built on a Blue Canyon XB1 bus 
https://ntrs.nasa.gov/api/citations/20240006994/downloads/SSC24_Starling_Swarm_Flight_Results.pdf


A typical 6U CubeSat is about 20 × 10 × 34 cm
https://www.nanosats.eu/cubesat#:~:text=6U%20CubeSat%20is%2020%20cm,12U%20as%20of%202019%20January.


--> Rough cross-sectional area ≈ 0.20 m × 0.34 m ≈ 0.068 m² on one face.

6U cubesat is about 12 kg
https://explorers.larc.nasa.gov/APMIDEX2016/MO/pdf_files/12-6U_CDS_2016-05-19_Provisional.pdf


Atmospheric Density models --> Vallado Table 8.4




\subsection{Conclusion}
This is the last section of your documentation. We are interested in reading on the
following topics:
a) Self-contained short summary of the project
b) Critical assessment of results
c) What have you learned? Lessons learned
d) Do you have ideas for potential continuations of this project?