\section{Problem Set 3}
\subsection{We are Close in Near-Circular Orbits}

\subsubsection{Initial Conditions for HCW}

We would like to use the Hill-Clohessy-Wiltshire (HCW) equations to give us a clear solution to the relative motion of the deputy satellites. However, the HCW equations assume that the motion of the depity with respect to the chief is very small compared to the orbit radius, as they rely on linearizing the non-linear equations of motion the The initial conditions for this are built over the original absolute and relative initial conditions for SV1, SV2, and SV3 that were previous defined in Table \ref{tab:abs_oe} and Table \ref{tab:relative_oe}.

The eccentricity of SV1 is already very low, as seen in Table \ref{tab:abs_oe_kepler} (and meets the conditions for applying HCW). Therefore, the orbital elements of the chief did not need to be modified. However, the relative orbital elements initial conditions create a large separation between SV1 and SV2/SV3. So the quasi-nonlinear relative orbital elements are modified to be as in Table \ref{tab:relative_oe_hcw} below. These are written in order as described in Equation \ref{eq:quasi_nonsign_roe}.

\begin{table}[h!]
\centering
\begin{tabular}{ll}
\toprule
\textbf{ID} & \textbf{HCW Conditions} \\
\midrule
SV2 & $\delta\boldsymbol{\alpha} = [0, 0, 0, 100, 0, 1000]~\text{m}$ \\
SV3 & $\delta\boldsymbol{\alpha} = [0, 0, 0, 200, 0, 800]~\text{m}$ \\
\bottomrule
\end{tabular}
\caption{Quasi-Nonsingular Relative Orbit Parameters for SV2 and Sv3 to apply for HCW}
\label{tab:relative_oe_hcw}
\end{table}

The main differences in these quasi-nonsingular relative orbital elements (and the justification for these changes), are:
\begin{itemize}
    \item $\delta\alpha$ is 0 for both SV2 and SV3, so that they have the same semi-major axes as SV1.
    \item The $\delta\lambda$ for both deputy satellites is reduced to 0. This helps reduce the separation between the deputies and the chief
    \item $\delta e_x$and $\delta i_x$ are set to zero for convenience, and to easily create a $\boldsymbol{e}-\boldsymbol{i}$ vector angle separation of $0^\circ$.
    \item $\delta e_y$ and $\delta i_y$ are set to convenient values close to the original relative orbital elements in Table \ref{tab:relative_oe}.
\end{itemize}

Based on these initial conditions, we see that the ratio of the maximum separation between spacecraft is small relative to the distance of the chief from the primary attractor.

TODO: Add something here to show this.

\subsubsection{Transforming the Initial Conditions} \label{sec:hcw_initial_conditions}
We can convert the initial conditions set in Table \ref{tab:relative_oe_hcw} to different representations. The initial conditions for the chief are not recalculated, as these have not been modified from previous sections.

\textbf{ECI and Absolute Orbital Elements} \\
Using the transformations highlighted in Equation \label{eq:quasi_nonsign_roe}, 
we convert the quasi-nonsingular relative orbital elements into absolute orbital elements. The results for SV2 and SV3 are given in Table \ref{tab:abs_oe_kepler_SV2_HCW} and Table \ref{tab:abs_oe_kepler_SV3_HCW}. The absolute orbital elements of the chief remain the same as in Table \ref{tab:abs_oe_kepler}.

TODO: Need to update these tables.

\begin{table}[h]
\centering
\begin{tabular}{cccccc} \hline
    $a$ & $e$ & $i$ & $\omega$ & $\Omega$ & $\nu$ \\ \hline 
     6944 km & 0.0016 & 99.4 $^\circ$ & 91.432$^\circ$ & -151.1$^\circ$ & -139.45$^\circ$ \\ \hline
\end{tabular}
\caption{Initial Keplerian Orbit Parameters of SV2, modified for applying HCW}
\label{tab:abs_oe_kepler_SV2_HCW}
\end{table}

\begin{table}[h]
\centering
\begin{tabular}{cccccc} \hline
    $a$ & $e$ & $i$ & $\omega$ & $\Omega$ & $\nu$ \\ \hline 
     6944 km & 0.0016 & 99.4 $^\circ$ & 91.432$^\circ$ & -151.1$^\circ$ & -139.45$^\circ$ \\ \hline
\end{tabular}
\caption{Initial Keplerian Orbit Parameters of SV3, modified for applying HCW}
\label{tab:abs_oe_kepler_SV3_HCW}
\end{table}

These Keplerian orbital elements are then converted to ECI co-ordinates using the expressions provided in Section \ref{sec:initial_ECI}. The initial ECI co-ordinates of the chief remain the same as in Equation \ref{eq:SV1_initial_ECI}. Equations \ref{eq:SV2_HCW1_ECI_initial} and \ref{eq:SV3_HCW1_ECI_initial} provide the ECI co-ordinates of SV2 and SV3. 


\begin{align} \label{eq:SV2_HCW1_ECI_initial}
    r_{0, ECI, SV2} &= \begin{bmatrix}
        -3091.3 \\
        -2937.0 \\
        -6503.8
    \end{bmatrix} km \\
    v_{0, ECI, SV2} &= \begin{bmatrix}
        -5.0008 \\
        -2.0031 \\
        4.0106
    \end{bmatrix} \frac{km}{s}
\end{align}


\begin{align} \label{eq:SV3_HCW1_ECI_initial}
    r_{0, ECI. SV3} &= \begin{bmatrix}
        -3091.4 \\
        -2937.0 \\
        -6503.7
    \end{bmatrix} km \\
    v_{0, ECI, SV3} &= \begin{bmatrix}
        -5.0009 \\
        -2.0029 \\
        4.0107
    \end{bmatrix} \frac{km}{s}
\end{align}


\textbf{Relative Position and Velocity in Chief's RTN frame, Orbital Element Differences} \\

Since the initial conditions of SV1, SV2, and SV3 are known, we can find the positions and velocities of SV2 and SV3 in SV1's RTN frame using the equations highlighted in Section \ref{sec:nonlinear_rel_eom}. These are provided below in Equations \ref{eq:SV2_HCW_RTN_init} and \ref{eq:SV3_HCW_RTN_init}.

\begin{align} \label{eq:SV2_HCW_RTN_init}
    r_{0, RTN. SV2} &= \begin{bmatrix}
        0.3256 \\
        -0.4989 \\
        -0.5971
    \end{bmatrix} km \\
    v_{0, RTN, SV2} &= \begin{bmatrix}
        -2.0102 \\
        -5.8056 \\
        -7.7372
    \end{bmatrix}\cdot 10^{-4} \frac{km}{s}
\end{align}

\begin{align} \label{eq:SV3_HCW_RTN_init}
    r_{0, RTN. SV2} &= \begin{bmatrix}
        0.2433 \\
        -0.3730 \\
        -0.4914
    \end{bmatrix} km \\
    v_{0, RTN, SV2} &= \begin{bmatrix}
        -1.5210 \\
        -4.3352 \\
        -6.3669
    \end{bmatrix}\cdot 10^{-4} \frac{km}{s}
\end{align}

The differences in the initial orbital elements between the chief and the deputies is best conveyed by the quasi-nonsingular relative orbital elements that are provided in Table \ref{tab:relative_oe_hcw}.

\subsubsection{Computing the Integration Constants}

The Hill-Clohessy Wiltshire equations for linearized relative orbital dynamics allows for analytical solutions of the RTN position $[x, y, z]^\top$ and the RTN velocity $[\dot{x}, \dot{y}, \dot{z}]^\top$ of the deputy satellites in the chief's RTN frame. This solution can be expressed as a function of time, when six integration constants are known (one for each state). The integration constants and the satellite's RTN state are related the Matrix-Vector solution for HCW, given in Equation \ref{eq:HCW_solution}, 


\begin{align} \label{eq:HCW_solution}
\begin{bmatrix}
x \\ y \\ z \\ \dot{x} \\ \dot{y} \\ \dot{z}
\end{bmatrix}
&=
\begin{bmatrix}
a I_{3 \times 3} & 0_{3 \times 3} \\
0_{3 \times 3} & a n I_{3 \times 3}
\end{bmatrix}
\begin{bmatrix}
1 & \sin nt & \cos nt & 0 & 0 & 0 \\
-\frac{3}{2}nt & 2 \cos nt & -2 \sin nt & 1 & 0 & 0 \\
0 & 0 & 0 & 0 & \sin nt & \cos nt \\
0 & \cos nt & -\sin nt & 0 & 0 & 0 \\
-\frac{3}{2} & -2 \sin nt & -2 \cos nt & 0 & 0 & 0 \\
0 & 0 & 0 & 0 & \cos nt & -\sin nt
\end{bmatrix}
\begin{bmatrix}
K_1 \\ K_2 \\ K_3 \\ K_4 \\ K_5 \\ K_6
\end{bmatrix},
\end{align}

where $a$ represents the semi-major axis, $n$ represents the mean motion, and $t$ is time since the initial conditions. $K_1$ through $K_6$ are the integration constants. THe matrix relating the integration constants and the state is called the State Transition Matrix (STM).

To calculate the integration constants, $t = 0$ in the STM, and the state is set to the initial conditions calculated in Section \ref{sec:hcw_initial_conditions}. Then the inverse of the STM matrix is taken to find the state.

\begin{align}
    \begin{bmatrix}
K_1 \\ K_2 \\ K_3 \\ K_4 \\ K_5 \\ K_6
\end{bmatrix} = \left(\begin{bmatrix}
a I_{3 \times 3} & 0_{3 \times 3} \\
0_{3 \times 3} & a n I_{3 \times 3}
\end{bmatrix}
\begin{bmatrix}
1 & 0 & 1 & 0 & 0 & 0 \\
0 & 2 & 0 & 1 & 0 & 0 \\
0 & 0 & 0 & 0 & 0 & 1 \\
0 & 1 & 0 & 0 & 0 & 0 \\
-\frac{3}{2} & 0 & -2 & 0 & 0 & 0 \\
0 & 0 & 0 & 0 & 1 & 0
\end{bmatrix}\right)^{-1} \begin{bmatrix}
x_0 \\ y_0 \\ z_0 \\ \dot{x}_0 \\ \dot{y}_0 \\ \dot{z}_0
\end{bmatrix}
\end{align}

For our chosen initial conditions, the integration constants computer are provided in Table \ref{tab:integration_constants_HCW}.

\begin{table}[ht]
    \centering
    \renewcommand{\arraystretch}{1.2}
    \begin{tabular}{c c c}
        \toprule
        \textbf{Constant} & \textbf{SV2} & \textbf{SV3} \\
        \midrule
        $K_1$ & $3.652\cdot10^{-7}$ & $2.764\cdot10^{-7}$ \\
        $K_2$ & $-3.848\cdot10^{-5}$ & $-2.886\cdot10^{-5}$ \\
        $K_3$ & $4.242\cdot10^{-5}$& $3.181\cdot10^{-5}$\\
        $K_4$ & $-2.076\cdot10^{-7}$ & $-1.583\cdot10^{-7}$ \\
        $K_5$ & $-1.0734\cdot10^{-4}$ & $-8.834\cdot10^{-5}$ \\
        $K_6$ & $-9.692\cdot10^{-5}$ & $-7.976\cdot10^{-5}$ \\
        \bottomrule
    \end{tabular}
    \caption{Integration Constants for SV2 and SV3 Used in HCW Analytical Solution}
    \label{tab:integration_constants_HCW}
\end{table}

\subsubsection{Relative State Propagation Using HCW Solution}

With the integration constants known, we can find the state (position and velocity) of SV2 and SV3 over all the 15 orbits we want to simulate, using the relation in Equation \ref{eq:HCW_solution}. Since this assumes circular orbits, we can use time as our independent variable rather than true anomaly.



\subsubsection{Analysis of HCW Solution Behavior}

TODO

\subsection{We are Close in Eccentric Orbits}